\documentclass[12pt]{article}
\usepackage[spanish]{babel}
\usepackage{natbib}
\usepackage{url}
\usepackage[utf8x]{inputenc}
\usepackage{amsmath}
\usepackage{graphicx}
\graphicspath{{images/}}
\usepackage{parskip}
\usepackage{fancyhdr}
\usepackage{vmargin}
\setmarginsrb{3 cm}{2.5 cm}{3 cm}{2.5 cm}{1 cm}{1.5 cm}{1 cm}{1.5 cm}

\title{Movimiento browniano}				% Title
\author{Tomás Israel Jasso Ramírez}			% Author
\date{\today}								% Date

\makeatletter
\let\thetitle\@title
\let\theauthor\@author
\let\thedate\@date
\makeatother

\pagestyle{fancy}
\fancyhf{}
\rhead{\theauthor}
\lhead{\thetitle}
\cfoot{\thepage}

\begin{document}

%%%%%%%%%%%%%%%%%%%%%%%%%%%%%%%%%%%%%%%%%%%%%%%%%%%%%%%%%%%%%%%%

%\tableofcontents
%\pagebreak

%%%%%%%%%%%%%%%%%%%%%%%%%%%%%%%%%%%%%%%%%%%%%%%%%%%%%%%%%%%%%%%%

\section{Introducción}
El movimiento browniano surgió a partir del estudio de las partículas de polen en el agua y desde entonces es utilizado en distintas áreas por su capacidad para demostrar y analizar sistemas con perturbaciones aleatorias. El movimiento browniano es un proceso continuo y adaptado definido en algún espacio de probabilidad. \cite{Leon}

\section{Metodología}


%\pagebreak
\bibliographystyle{plain}
\bibliography{biblist}

\end{document}
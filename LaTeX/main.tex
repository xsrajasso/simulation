\documentclass[12pt]{article}
\usepackage[spanish]{babel}
\usepackage{natbib}
\usepackage{url}
\usepackage[utf8x]{inputenc}
\usepackage{amsmath}
\usepackage{graphicx}
\graphicspath{{images/}}
\usepackage{parskip}
\usepackage{fancyhdr}
\usepackage{vmargin}


\title{Definición de sistemas digitales}								% Title
\author{Tomas Israel Jasso Ramírez}			% Author
\date{\today}								% Date

\makeatletter
\let\thetitle\@title
\let\theauthor\@author
\let\thedate\@date
\makeatother

\pagestyle{fancy}
\fancyhf{}
\rhead{\theauthor}
\lhead{\thetitle}
\cfoot{\thepage}

\begin{document}

%%%%%%%%%%%%%%%%%%%%%%%%%%%%%%%%%%%%%%%%%%%%%%%%%%%%%%%%%%%%%%%%%%%%%%%%%%%%%%%%%%%%%%%%%
\begin{titlepage}

%LOGOS
\begin{tabular}{p{0.62\textwidth} p{0.75\textwidth}}
      \includegraphics[width=0.25\textwidth]{UANL.jpeg} & \includegraphics[trim= 0 110 0 0 ,clip,width=0.35\textwidth]{FIME.png}
\end{tabular}

\begin{center}
%UANL-FIME
\par \vspace{1cm}
{\Large\textbf{Universidad Autonóma de Nuevo León \\*[0.2cm] Facultad de Ingeniería Mecánica y Eléctrica}}

\par\vspace{1.5cm}{
\LARGE\textbf{\#Electrónica Digital}}

\par\vspace{0.7cm}{
\Large\textbf{\#Definición de sistemas digitales}}

\par\vspace{2cm}{
\large{*Nombre o nombres de los integrantes junto a su matricula}}

\par\vspace{.5cm}{
\large\textbf{\#Tomas Israel Jasso Ramírez \hspace{3.5cm} \#1943530}}

\par\vspace{1cm}{
\large\textbf{\#Jesús Daniel Garza Camarena}}

\par\vspace{1cm}{
\large\textbf{Semestre Agosto-Enero 2020-2021}}

\begin{flushright}
\par\vspace{1cm}{
\large\textbf{\#Jueves V1-V3}}
\end{flushright}

\end{center}

\par\vspace{1cm}{
\large\textbf{San Nicolás de los Garza, N.L. \hspace{3cm} \#07/09/2020}}


\end{titlepage}

%%%%%%%%%%%%%%%%%%%%%%%%%%%%%%%%%%%%%%%%%%%%%%%%%%%%%%%%%%%%%%%%%%%%%%%%%%%%%%%%%%%%%%%%%

%\tableofcontents
%\pagebreak

%%%%%%%%%%%%%%%%%%%%%%%%%%%%%%%%%%%%%%%%%%%%%%%%%%%%%%%%%%%%%%%%%%%%%%%%%%%%%%%%%%%%%%%%%

\section{Actividad Formativa N°2}
\vspace{1cm}
{\large \textbf{Objetivo:}} Conocer y entender el contexto de la materia con la que se trabajará

\vspace{.35cm}

\large\textbf{1. Definición Sistemas Digitales} \\
\normalsize A \textbf{digital system} is a combination of devices designed to manipulate logical information or physical quantities that are represented in digital form; that is, the quantities can take on only discrete values." \cite{tocci}

\vspace{.35cm}

\large\textbf{2. Definición Sistemas Digitales} \\
\normalsize "Digital systems use discrete quantities to represent information. Discrete means distinct or separated as opposed to continous or connected" \cite{godse} 

\vspace{.35cm}

\large\textbf{3. Definición Diseño Combinacional} \\
\normalsize "Un sistema combinacional es aquel donde los valores de salida dependen únicamente de las combinaciones de entrada. En este sistema el número de entradas puede ser mayor, menor o igual al número de salidas." \cite{garza}

\vspace{.35cm}

\large\textbf{4. Definición Diseño Secuencial} \\
\normalsize "Un circuito secuencial se especifica con una sucesión temporal de entradas, salidas y estados internos. [...] El cambio de estado interno se da cuando hay un cambio en las variables de entrada." \cite{mano}

\vspace{.35cm}

\large \textbf{Conclusión} \\
\normalsize Los temas de electrónica digital se manejan en términos discretos para sus distintos diseños y tienen manipulaciones lógicas que normalmente se trabajan a través de relojes.




\newpage
\bibliographystyle{ieeetr}
\bibliography{biblist}

\end{document}